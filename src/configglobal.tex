% As configurações gerais são colocadas aqui, como novos comandos para o corpo do texto,
% informações de bookmark para o PDF, tamanho de parágrafos, entre outros.

% ----------------------------------------------------------
% Configurações do pacote BACKREF
% ----------------------------------------------------------
% Usado sem a opção hyperpageref de backref
\renewcommand{\backrefpagesname}{Citado na(s) página(s):~}
% Texto padrão antes do número das páginas
\renewcommand{\backref}{}
% Define os textos da citação
\renewcommand*{\backrefalt}[4]{
	\ifcase #1 %
		Nenhuma citação no texto.%
	\or
		Citado na página #2.%
	\else
		Citado #1 vezes nas páginas #2.%
	\fi}%
% ---

% ----------------------------------------------------------
% \theoremstyle{plain}
\newtheorem{theorem}{Teorema}%[chapter]
\newtheorem{lemma}{Lema}%[chapter]
\providecommand*{\lemmaautorefname}{Lema}
\newtheorem{proposition}{Proposition}%[chapter]
\providecommand*{\propositionautorefname}{Proposition}
\newtheorem{corollary}{Corollary}%[chapter]
\providecommand*{\corollaryautorefname}{Corollary}
\newtheorem{conjecture}{Conjecture}%[chapter]
\providecommand*{\conjectureautorefname}{Conjecture}
\newtheorem{definition}{Definition}%[chapter]
\providecommand*{\definitionautorefname}{Definition}
\newtheorem{notation}{Notation}%[chapter]
\providecommand*{\notationautorefname}{Notation}
\newtheorem{remark}{Observation}%[chapter]
\providecommand*{\remarkautorefname}{Observation}
\newtheorem{example}{Example}%[chapter]
\providecommand*{\exampleautorefname}{Example}
\newtheorem{note}{Note}%[chapter]
\providecommand*{\noteautorefname}{Note}
\newtheorem{algo}{Algorithm}
% ----------------------------------------------------------

% ----------------------------------------------------------
% Conjunto de configuracoes para o pacote 'listings'
% ----------------------------------------------------------
\lstset{
  language=C++,
  basicstyle=\ttfamily, 
  keywordstyle=\color{blue}, 
  stringstyle=\color{verde}, 
  commentstyle=\color{red}, 
  extendedchars=true, 
  showspaces=false, 
  showstringspaces=false,
  numbers=left,
  numberstyle=\tiny,
  breaklines=true, 
  backgroundcolor=\color{green!10},
  breakautoindent=true,
  fontadjust=false
}
% ----------------------------------------------------------

% ----------------------------------------------------------
% Informações do PDF
% ----------------------------------------------------------
% Configurações de aparência do PDF final
% ---
% alterando o aspecto da cor azul
\definecolor{blue}{RGB}{41,5,195}
\definecolor{verde}{rgb}{0,0.5,0}
% ---
\makeatletter
\hypersetup{
  %pagebackref=true,
  pdftitle={\@title},
  pdfauthor={\@author},
  pdfsubject={%
    \imprimirtipotrabalho\ apresentada ao Instituto de Matemática, Estatística %
    e Computação Científica da Universidade Estadual de Campinas como parte dos %
    requisitos exigidos para a obtenção do título de \imprimirtitulacao\ em %
    \imprimircurso.
  },
  pdfcreator={LaTeX with unicamp-abnTeX2},
  pdfkeywords={abnt}{latex}{abntex}{abntex2}{trabalho acadêmico},
  colorlinks=true,		% false: boxed links; true: colored links
  linkcolor=blue,		% color of internal links
  citecolor=blue,		% color of links to bibliography
  filecolor=magenta,		% color of file links
  urlcolor=blue,		% color of internet links
  bookmarksdepth=4
}
\makeatother
% ----------------------------------------------------------

% ----------------------------------------------------------
% COMANDOS GLOBAIS
% ----------------------------------------------------------
\everymath{\displaystyle}
% ---
\renewcommand{\sin}{\mathrm{sen}}
\renewcommand{\tan}{\mathrm{tg}}
\renewcommand{\csc}{\mathrm{cossec}}
\renewcommand{\cot}{\mathrm{cotg}}
% ---
\DeclareMathOperator{\posto}{\mathrm{posto}}
\DeclareMathOperator{\conv}{\mathrm{conv}}
\DeclareMathOperator{\diag}{\mathrm{diag}}
\DeclareMathOperator{\argmin}{\mathrm{arg}\min}
\DeclareMathOperator{\argmax}{\mathrm{arg}\max}
% ---
% O tamanho do parágrafo é dado por:
\setlength{\parindent}{2.0cm}
% Controle do espaçamento entre um parágrafo e outro:
\setlength{\parskip}{0.2cm}  % tente também \onelineskip
% ---
% Para que apareça o nome 'Capítulo X' antes do título de cada capítulo
% \chapterstyle{default}
% ----------------------------------------------------------
\newsubfloat{figure}% Allow subfloats in figure environment
\providecommand*{\subfigureautorefname}{Subfigura}

% ==================================================
% COMANDOS DO BIBI
% ==================================================
\newcommand{\mylipsum}[1]{\textcolor{black!50}{\lipsum[#1]}}