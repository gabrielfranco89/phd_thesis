% -------------------------------------------------------------
%  RESUMOS
\setlength{\absparsep}{18pt} % ajusta o espaçamento dos parágrafos do resumo
% -------------------------------------------------------------
% ATENÇÃO: o ambiente 'otherlanguage*' deve ser usado para o resumo que não está na
% língua vernácula do trabalho, com a respectiva opção linguística do pacote 'babel'.
% -------------------------------------------------------------
% resumo em PORTUGUÊS (OBRIGATÓRIO)
\begin{resumo}[Resumo]
  \begin{otherlanguage*}{brazil}
    Em um cenário de aquecimento global, é interessante adotar medidas de eficiência na produção de energia elétrica. Para isso, monitorar o consumo de diferentes tipos de consumidores é essencial para entender as possível demandas e consequentes tomadas de decisão para modelos energéticos mais eficientes. No entanto, é raro observar o consumo de cada indivíduo em uma frequência adequada para realizar análises estatísticas, tendo na maioria das vezes apenas acesso ao consumo acumulado nas unidade de distribuição de energia. A metodologia proposta nesta tese separa a curva de consumo agregada nessas unidades em curvas típicas para cada indivíduo abastecido através de um completo modelo funcional agregado, utilizando um modelo linear e expansão por funções bases na tarefa de desagregação e construindo uma detalhada estrutura de covariância dos dados. Além disso, é também proposto um modelo de variáveis latentes para agrupar as unidades de distribuição com curvas típicas similares. O modelo é testado em uma série de experimentos com dados simulados e aplicado a um conjunto de dados de estações de energia elétrica no Reino Unidos. 

    \textbf{Palavras-chave}: Separação de sinal, modelo funcional, modelo funcional agregado, processos gaussianos, expansão por funções base.
 \end{otherlanguage*}
\end{resumo}
% -------------------------------------------------------------
% -------------------------------------------------------------
% resumo em INGLÊS (OBRIGATÓRIO)
\begin{resumo}[Abstract]
 \begin{otherlanguage*}{english}
    The global warmth alert pushed organizations worldwide to spend efforts on efficient energy programs. Load monitoring is an important task to understand electrical demand and provide information for future power network plans. However, observe the individual consumption is still rare besides the growing of smart meters technology, thus it might be interesting to gather information of electrical supplier units and make decisions based on the estimated information of individual customers. The proposed methodology separates the aggregated load in energy suppliers  into typical curves for each supplied customer. The approach supposes a Gaussian process and model the average load response via a linear combination of expanded mean curves, making possible the addition of explanatory variables and an additional functional component to typical curve response. Moreover, we propose a model-based clustering for supplier units with similar separated curves. The model is implemented in  series of experiments with simulated data to access model performance and applied to real data of load monitoring of stations in United Kingdom.

    \textbf{Keywords}: Blind source separation, functional aggregated model, gaussian process, basis function expansion.
 \end{otherlanguage*}
\end{resumo}
% -------------------------------------------------------------