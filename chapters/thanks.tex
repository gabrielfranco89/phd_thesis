  A special thanks to professor Gavin Shaddick whom provided the dataset used in this thesis.
  The following thanks will be addressed in Portuguese.

  Primeiramente, agradeço à Coordenação de Aperfeiçoamento de Pessoal de Nível Superior (CAPES) pelo apoio e confiança em meu trabalho através da bolsa de fomento à pesquisa durante todo meu doutorado e boa parte do mestrado.

  Em seguida, agradeço ao excelente trabalho de orientação das professoras Nancy Garcia e Camila de Souza. A paciência, atenção e dedicação dispendidas neste trabalho foram essenciais para guiar essa pesquisa tão interessante e iluminar meus caminhos nesta jornada. Mesmo com muitas outras tarefas administrativas e didáticas, além do cenário de pandemia no ano de 2020, jamais me faltou amparo, apoio e mentoria. São dois exemplos de mentoras que quero guardar para o resto da minha vida e sonho um dia conseguir exercer tal função com tamanha propriedade. Espero que cada pessoa que eu possa vir a orientar no futuro tenha a mesma segurança e confiança que eu tive durante esse período sob a tutela dessas duas excelentes profissionais e pessoas maravilhosas.

  Em especial, dedico essa tese e toda minha carreira aos meus pais. Tive o privilégio de contar com o apoio e amor incondicionais que apenas essas duas pessoas são capazes de prover. Agradeço desde o apoio e investimento no período de vestibular, a confiança e apoio ao me ver saindo de casa rumo à Campinas e principalmente o apoio à minha opção de carreira. A todo momento é possível sentir a admiração e o orgulho de ambos em cada conquista e espero que essa seja mais uma para se orgulharem. Newton disse ``sobre os ombros de gigantes'', eu digo que os gigantes que tornaram possível visualizar e buscar esse horizonte foram meus pais.

  Também foi fundamental o apoio dos ciclos de amizade que fizeram parte desta jornada. Amigos de longa data que me apoiam desde a graduação como Diego e Joyce Fujiwara e Renato Reis. Destes mais antigos, tenho um obrigado especial ao Darcy Camargo, meu amigo, meu colega e meu irmão. Além de ser uma ótima fonte de conhecimento para tirar minhas dúvidas em matemática e probabilidade, foi e é companheiro de todas as horas, uma pessoa de confiança e a quem dedico boa parte do meu sucesso. Ainda pelo IMECC, agradeço ao acolhimento de todas as pessoas maravilhosas que circulavam pela sala 1B do prédio anexo da pós-graduação. Era sempre uma alegria subir as escadas para receber e doar alegria ao redor de um café.

  Aproveito para dedicar uma parte desses agradecimentos a pessoas que me inspiraram ao longo desses anos a ser um melhor indivíduo e profissional, seja diretamente ou apenas pelo fato de permitir que eu pudesse acompanhar suas carreiras: obrigado à Ligia Steinberg, Charles Almeida, Leonora Cardani, Helen Silva, Victor Freguglia, Ana Fontenelle e Thays Gomes.

  Não posso deixar de agradecer ao Grupo Ginástico Unicamp, onde tenho amizades que guardo com muito carinho e onde sempre fui muito bem recebido e acolhido. O grupo proporcionou experiências além da grade curricular que vou guardar sempre com muito carinho e me apresentou a pessoas que hoje fazem parte da minha vida além da Unicamp.

  Por fim, um imenso agradecimento a todos os professores e professoras que passaram pela minha vida e também a todos e todas do departamento de estatística do Instituto de Matemática, Estatística e Computação Científica, nosso IMECC. Em especial, minhas primeiras orientadoras Mariana Motta e Hildete Pinheiro, meu orientador de mestrado Aluísio Pinheiro, à professora Verónica González-López que me recebeu de braços abertos no curso de estatística. A tese que aqui escrevo é também fruto de um esforço coletivo de todos vocês. % apesar de algumas pessoas

  E agredeço a você que está lendo esta tese. Seu interesse nesta pesquisa é minha maior recompensa.

  Muito obrigado, pessoas!

